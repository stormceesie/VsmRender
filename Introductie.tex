\section{Introductie}


\subsection{Aanleiding en Context}

Voortman Machinery, gevestigd in Rijssen is producent van geavanceerde metaalbewerkingsmachines. Voortman is opgericht in 1968 \cite{web:VoortmanGeschiedenis}. In het begin richtte Voortman zich vooral op de productie van staalconstructies. In 1995 begon Voortman ook staalbewerkingsmachines te ontwikkelen en te produceren waaruit de naam Voortman Steel Machinery is ontstaan. In de loop der jaren is het bedrijf uitgegroeid tot een wereldwijde fabrikant van \gls{cnc}-staalbewerkingsmachines met hierbij ook de bijpassende softwareoplossingen.

\vspace{1cm}

 In veel machines van Voortman zitten spindels, deze spindels worden gebruikt in de machines om bijvoorbeeld te vrezen, tapen of boren. Na een bepaalde tijd kunnen deze spindels door slijtage of schade niet meer correct functioneren. Deze spindels kunnen in veel gevallen nog gereviseerd worden, dit kan echter veel geld kosten wanneer deze worden opgestuurd naar de fabrikant.
 
 \vspace{1cm}
 
  Voortman wil daarom de spindels zelf kunnen reviseren. Echter ondersteunt de huidige testkast (figuur \ref{fig:TestKastFoto}) bij Voortman slechts één soort spindel terwijl er wel veel verschillende soorten gebruikt worden in het machine portfolio van Voortman. Hieruit kwam de opdracht om de testkast aan te passen zodat zo veel mogelijk soorten spindels kunnen worden getest op de testkast van Voortman zelf.

\begin{figure}[h]
	\centering
	\includegraphics[width=300pt]{TestKast}
	\label{fig:TestKastFoto}
	\caption{Foto van de bestaande testkast}
\end{figure}

\newpage

\subsection{Doelstelling}

Op het moment zit er een Beckhoff \gls{AX5140} servo drive in de testkast gebouwd voor het aansturen van de motoren. De parameters die in deze drive zitten kunnen momenteel alleen de servomotor HQL100X aansturen. Andere motoren kunnen niet worden aangestuurd met deze parameters omdat deze motoren bijvoorbeeld een andere encoder hebben, andere \gls{pid}-instellingen of de motor werkt simpelweg volgens een ander principe (asynchrone of synchrone servo). De \gls{AX5140} heeft meer dan 400 parameters de meeste hiervan kunnen worden aangepast.

\vspace{1cm}

Ook is het testprotocol van Voortman op de spindels niet erg geavanceerd zo worden motoren alleen aangestuurd naar een bepaald tourental voor een bepaald aantal seconden waarbij er gelet wordt of de motor niet te veel trilt en of deze niet te warm wordt. Zij kwantificeren deze waardes niet het is daarom erg lastig om aantoonbaar de motor te beoordelen op zijn prestaties.

\vspace{1cm}

De wens is daarom vanuit Voortman om uit te zoeken welke parameters relevant zijn voor wanneer er een motor gewisseld wordt en welke waardes dit moeten zijn voor de motoren die zij willen gaan testen. Met daarbij een programma die ervoor kan zorgen dat deze parameters eenvoudig naar de drive kunnen worden geschreven. Daarnaast moet er uitgezocht worden wat Voortman wil gaan testen aan de motoren en wat hiervoor eventueel voor nodig is zoals een trilling sensor die de trillingen van de spindel kan meten.

\newpage

\subsection{Onderzoeksvragen}

Tijdens de afstudeerperiode zullen de volgende onderzoeksvragen beantwoord worden:

\begin{enumerate}
	\item Welke specifieke motorparameters zijn noodzakelijk om minimaal 90\% van de door Voortman gebruikte motortypes aan te sturen en hoe kunnen deze parameters binnen zes weken worden geïdentificeerd?
	
	\item Hoe kunnen motorparameters automatisch geschreven worden naar de drive, en hoe kan dit binnen zes weken worden geïmplementeerd en getest voor tenminste drie motortypes?
	
	\item Welke encoderinterfaces (e.g. SSI of EnDat) zijn noodzakelijk om 90\% van de door Voortman gebruikte motoren te ondersteunen op de testkast, en hoe kunnen deze interfaces binnen zes weken worden getest op te testkast.
	
	\item Welke meetmethodes en testcriteria (e.g. temperatuur, stroomverbruik toerental en trillingen) kunnen worden gebruikt om minimaal 90\% van de potentiële defecten in gereviseerde motoren te detecteren, en hoe kan een testprotocol binnen acht weken worden opgesteld en gevalideerd?
	
	\item Hoe kan de software op de testkast binnen acht weken worden geprogrammeerd om automatisch testresultaten te loggen, waarbij afwijkingen in bijvoorbeeld trillingen of overbelasting kunnen worden gedetecteerd en gerapporteerd in een gestructureerd PDF-rapport.
	
	\item Welke software of hardware aanpassingen zijn er nodig om binnen drie maanden een modulaire testomgeving te onwikkelen, waarbij nieuwe motortypes zonder codewijzigingen en een maximale configuratie tijd van maximaal één uur kunnen worden toegevoegd?
\end{enumerate}

\newpage

\subsection{Afbakening}

