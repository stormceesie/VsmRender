\usepackage[a4paper, left=3.5cm,right=3.5cm,bottom=3.5cm]{geometry}
\usepackage[final]{graphicx}
\graphicspath{{../images/}}
\usepackage{mathrsfs}
\usepackage{newtxtext}
\usepackage{float}
\usepackage{comment}
\usepackage{tikz}
\usetikzlibrary{calc}
\usepackage{eso-pic}
\usepackage[dutch]{babel}
\usepackage{fancyhdr}
\usepackage{multirow}
\usepackage[table]{xcolor}
\usepackage[hidelinks]{hyperref}
\usepackage[toc, acronym, nomain]{glossaries-extra}
\usepackage{attachfile}
\usepackage{array}
\usepackage{xltabular}
\usepackage{hyperref}
\usepackage{float}
\usepackage{url}
\usepackage{xltabular}

% Definieer afkortingen
\newacronym{cnc}{CNC}{Computer Numerical Control}
\newacronym{pid}{PID}{Proportional-Integral-Derivative}
\newacronym{gui}{GUI}{Graphical User Interface}
\newacronym{usb}{USB}{Universal Serial Bus}
\newacronym{AX5140}{AX5140}{Digital Compact Servo Drive welke gemonteerd is in de testkast}

\newacronym{AX5801}{AX5801}{Veiligheidskaart; uitbreiding van de \gls{AX5140}}
\newacronym{AX5701}{AX5701}{Encoder option card; uitbreiding van de \gls{AX5140}}
\newacronym{AX5702}{AX5702}{Encoder option card; zelfde als \gls{AX5701} maar dan met een extra port}

\newacronym{STO}{STO}{Safe Torque Off}
\newacronym{SSI}{SSI}{Synchronous Serial Interface}
\newacronym{PDF}{PDF}{Portable Document Format}
\newacronym{NC}{NC}{Numerical Control}

\newacronym{DFT}{DFT}{Discrete Fourier Transform}
\newacronym{DTFT}{DTFT}{Discrete-time Fourier Transform}
\newacronym{FFT}{FFT}{Fast Fourier Transform}

\newacronym{BPFO}{BPFO}{Balls Pass Frequency Outer Race}
\newacronym{BPFI}{BPFI}{Ball Pass Frequency Inner Race}
\newacronym{BSF}{BSF}{Ball Spin Frequency}
\newacronym{FTF}{FTF}{Fundamental Train Frequency}

\newacronym{SS1}{SS1}{Safe Stop 1}

\newacronym{A}{A}{Ampère}
\newacronym{V}{V}{Voltage}

\newacronym{AC}{AC}{Alternating Current}
\newacronym{EEPROM}{EEPROM}{Erasable Programable Read-Only Memory}
\newacronym{FRAM}{FRAM}{Ferroelectric Random Access Memory}
\newacronym{SoE}{SoE}{\gls{SERCOS} over \gls{EtherCAT}}
\newacronym{SERCOS}{SERCOS}{SErial Real-time COmmunication System}

\newacronym{EtherCAT}{EtherCAT}{Ethernet for Control Automation Technology}
\newacronym{TwinCAT}{TwinCAT}{The Windows Control and Automation Technology}
\newacronym{MQTT}{MQTT}{Message Queuing Telementry Transport}
\newacronym{PLC}{PLC}{Programmable Logic Controller}
\newacronym{ADS}{ADS}{Automation Device Specification}
\newacronym{IDN}{IDN}{Identification Number}
\newacronym{WPF}{WPF}{Windows Presentation Foundation}
\newacronym{POC}{POC}{Proof Of Concept}
\newacronym{dB}{dB}{Decibel}

\newacronym{MCSA}{MCSA}{Motor Current Signature Analysis}

\newacronym{HMI}{HMI}{Human Machine Interface}

\makeglossaries

\usepackage{chngcntr}
\numberwithin{figure}{section}
\numberwithin{table}{section}

% Mooi hoekje links bovenin net als in de Voortman template
\newcounter{pagecount}
\AtBeginShipout{%
	\stepcounter{pagecount} % Tel de pagina's
	\ifnum\value{pagecount}>1 % Pas toe vanaf pagina 2
	\AtBeginShipoutUpperLeft{%
		\begin{tikzpicture}[remember picture, overlay]
			\fill[customred] (current page.north west) ++(3.0cm,0) -- (current page.north west) -- ++(0,-6.5cm) -- cycle;
		\end{tikzpicture}
	}
	\fi
}

% Makkelijk commando om eisen in te voegen met xltabular
\newcommand{\eistabel}[6]{
	\begin{xltabular}{\linewidth}{|p{0.15\linewidth}|p{0.5\linewidth}|p{0.2\linewidth}|}
		\caption{#1 #2}\\
		\hline
		\textbf{#1} & \textbf{#2} & \textbf{#3} \\
		\hline
		\endfirsthead
		\hline
		\textbf{#1} & \textbf{#2} & \textbf{#3} \\
		\hline
		\endhead
		\hline
		\endfoot
		\hline
		\endlastfoot
		\textbf{Stelling} & \multicolumn{2}{p{0.7\linewidth}|}{#4}\\
		\hline
		\textbf{Meetmethode} & \multicolumn{2}{p{0.7\linewidth}|}{#5} \\
		\hline
		\textbf{Opmerking} & \multicolumn{2}{p{0.7\linewidth}|}{#6} \\
		\hline
	\end{xltabular}
}

\pagestyle{fancy}

\fancyhf{}

\definecolor{customred}{HTML}{D78C91}

\newenvironment{myfont}{\fontfamily{phv}\selectfont}{\par}

\fancyhead[L]{\textbf{Afstudeerstage}}
\fancyhead[C]{}
\fancyhead[R]{\today}

% Voettekst-instellingen
\fancyfoot[L]{Auteur: Florent Kegler}
\fancyfoot[C]{Pagina \thepage}
\fancyfoot[R]{\includegraphics[width=4cm]{VoortmanLogo}}

% maak de tabel ietsje groter want de standaard is soms beetje smal
\setlength{\arrayrulewidth}{0.5mm}
\setlength{\tabcolsep}{18pt}
\renewcommand{\arraystretch}{1.5}