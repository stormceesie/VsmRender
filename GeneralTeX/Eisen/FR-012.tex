\eistabel{FR-012}{\gls{RMS} (Root Mean Squared) of the motor vibration}{GEMIDDELD}{De testkast moet aan kunnen geven of de spindel gemiddeld gezien significant meer trilt dan de referentie spindel.}{De eis is behaald wanneer de testkast de gemeten trillingen vergelijkt met de referentie spindel metingen. Dit kan worden berekend op de volgende manier:
	
	Om te weten of de spindel meer of minder trilt kan worden gebaseerd op de RMS-waarde van de sample de formule hiervoor is als volgt:
	
	\begin{equation}
		X_{RMS}=\sqrt{\frac{1}{n}\sum_{i=0}^{n-1}X_i^2}
	\end{equation}
	
	
	Deze waarde van de gereviseerde spindel mag niet te veel afwijken van de gemiddelde waarde van de referentiegroep (juist werkende spindels) +- 2x de standaarddeviatie hiervan. Waarbij de sample grootte van de referentiegroep ideaal >30 is om zo een normaalverdeling te hebben echter kan dit erg veel werk zijn een lagere sample grootte is daarom ook al waardevol.
	
	De volgende vergelijking kan gebruikt worden om de standaarddeviatie te berekenen van de controlegroep:
	
	\begin{equation}
		\sigma_{RMS}=\sqrt{\frac{1}{N}\sum_{i=0}^{N}\left(X_{i,RMS}-{\bar{X}}_{RMS}\right)^2}
	\end{equation}
	
	
	Hier kunnen we een getal mee berekenen die de gebruiker verteld hoeveel de $X_{RMS}$ van de geteste spindel van het gemiddelde zit van de controlegroep met de volgende formule:
	\begin{equation}
		\Delta X_{RMS}=\frac{X_{RMS,gereviseerd}-{\bar{X}}_{RMS}}{\sigma_{RMS}}
	\end{equation}
	 Wanneer de $\left|\Delta X_{RMS}\ \right|$ groter is dan 2 betekend dit dat het verschil significant is en dat er mogelijk een probleem is. Met uitzondering van wanneer $\Delta X_{RMS}$ negatief is dit zou namelijk betekenen dat de trilling minder is geworden ten opzichte van de control groep na het reviseren.
}{}