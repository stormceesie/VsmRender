\section{Electrisch Ontwerp}

Vanuit Voortman was het de wens om een trillingssensor toe te voegen (HR-002) aan het systeem zodat trillingen van de motor of de spindel gekwantificeert kunnen worden. Daarnaast is was er de wens om een luchtdruk sensor bij de testkast op te bouwen zodat de \gls{PLC} de spindel kan blokkeren op het moment dat er geen luchtdruk op de testkast staat. De toolwisselaar van de spindel moet namelijk altijd helemaal open staan op het moment dat de spindel gaat draaien anders kan dit voor schade zorgen (HR-001). Verder was er de wens om de temperatuur van de spindel te meten (HR-003) om eventuele heatspots te detecteren tijdens het testen. Tot slot is er de wens om de deurschakelaars op een eigen veiligheids ingang te plaatsen zodat bij het openen van de deuren de toolwisselaar nog wel werkt (FR-008, FR-018).

